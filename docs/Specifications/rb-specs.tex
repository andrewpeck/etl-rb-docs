% Created 2022-02-18 Fri 15:56
% Intended LaTeX compiler: pdflatex
\documentclass[11pt]{article}
             \usepackage[utf8]{inputenc}
             \usepackage[T1]{fontenc}
             \usepackage{fixltx2e}
             \usepackage{fullpage}
             \usepackage{graphicx}
             \usepackage{longtable}
             \usepackage{float}
             \usepackage{wrapfig}
             \usepackage{rotating}
             \usepackage[normalem]{ulem}
             \usepackage{amsmath}
             \usepackage{textcomp}
             \usepackage{marvosym}
             \usepackage{wasysym}
             \usepackage{amssymb}
             \usepackage{hyperref}
             %\usepackage{mathpazo}
             \renewcommand{\familydefault}{\sfdefault}
             \usepackage{color}
             \usepackage{enumerate}
             \definecolor{bg}{rgb}{0.95,0.95,0.95}
             \tolerance=1000
                          
             \linespread{1.1}
             \hypersetup{pdfborder=0 0 0}
\author{Andrew Peck, Daniel Spitzbart}
\date{\today}
\title{Readout Board Specifications}
\hypersetup{
 pdfauthor={Andrew Peck, Daniel Spitzbart},
 pdftitle={Readout Board Specifications},
 pdfkeywords={},
 pdfsubject={},
 pdfcreator={Emacs 28.0.91 (Org mode 9.6)}, 
 pdflang={English}}
\begin{document}

\maketitle

\section{Readout Board Specifications}
\label{sec:orge4fdbab}

Authors: Andrew Peck, Daniel Spitzbart

Modification Date: 2022-02-18 15:56

Status: This document is missing 60 pieces of information concerning  58 specifications
\begin{itemize}
\item Specifications are \textbf{-3.4\% complete}.
\end{itemize}

A pdf version of this document can be found \href{./rb-specs.pdf}{here}. Please check the timestamp to ensure it is up to date. The master copy of this document is an emacs org mode file found \href{https://gitlab.cern.ch/cms-etl-electronics/readout-board-docs/-/blob/master/docs/Specifications/rb-specs.org}{here}.

The latest RB schematic can be found here \url{https://gitlab.cern.ch/cms-etl-electronics/readout-board-pcb/uploads/183954f3a47f967752902acf8ae9c3d3/ETL\_RB\_V1.6.PDF}

\setcounter{tocdepth}{3}
\tableofcontents

\subsection{Description}
\label{sec:orge5c745c}

The readout board will be designed in 3 different flavors, called the RB-3, RB-6, and RB-7, where the suffix number represents the number of ``full-modules'' that the readout board services.

\begin{itemize}
\item An RB-3 will interface with 3 modules, meaning 12 ETROCs and 12 sensors.
\item An RB-6 will interface with 6 modules, meaning 24 ETROCs and 24 sensors.
\item An RB-7 will interface with 7 modules, meaning 28 ETROCs and 28 sensors.
\end{itemize}

The readout board consists of one or more lpGBTs, a GBT-SCA, a VTRX+, a number of linPOL12 regulators, and associated connectors / passive components required to interface with the external systems.

\begin{itemize}
\item Each RB will have 1 GBT-SCA
\item Each RB will have 2 lpGBTs
\item Each RB will have 1 VTRX+
\item Each RB will have 6 linPOL12s
\begin{itemize}
\item 2 for VTRX+ RX
\item 2 for VTRX+ TX
\item 1 for GBT-SCA analog power
\item 1 for GBT-SCA digital power
\end{itemize}
\end{itemize}
\subsection{Drawings}
\label{sec:org42ac7ca}
\subsection{Interfaces}
\label{sec:org0f96899}
\subsubsection{Power Board Interface}
\label{sec:orge05ed4e}
The interface to the power board will consist of:
\begin{itemize}
\item The power board interface will use connector part number \uline{UNKNOWN}.
\item The pinout of these connectors is \uline{UNKNOWN}.
\item The placement of these connectors is \uline{UNKNOWN}.
\item The quantity of these connectors is \uline{UNKNOWN}.
\end{itemize}
\subsubsection{Module Interface}
\label{sec:org8e69da1}
\begin{enumerate}
\item Mechanical Interface
\label{sec:orge8521ee}
\begin{itemize}
\item the module shall be aligned to the readout board using an \uline{UNKNOWN} keying mechanism
\end{itemize}
\item Signal Interface
\label{sec:org201ab19}
The signal interface to the module will consist of:
\begin{itemize}
\item The module will use connector part number \uline{UNKNOWN}.
\item The pinout of the module connectors is \uline{UNKNOWN}.
\item The placement of these connectors is \uline{UNKNOWN}.
\end{itemize}
\item BV Interface
\label{sec:orgc236ad4}
The signal interface to the module will consist of:
\begin{itemize}
\item The BV to module interface will use connector part number \uline{UNKNOWN}.
\item The pinout of these connectors is \uline{UNKNOWN}.
\item The placement of these connectors is \uline{UNKNOWN}.
\end{itemize}
\end{enumerate}
\subsubsection{Fiber Optic Interface}
\label{sec:org3ff176a}
The fiber optic interface to CMS is through a VTRX+. The readout board will host both the VTRX+, as well as the MT Ferrule that is required to connect between the VTRX+ and a ``naked fanout''.
\begin{itemize}
\item The size of the cutout for the MT ferrule is \uline{UNKNOWN}.
\item The location of the cutout for the MT ferrule is:
\begin{itemize}
\item RB-3: \uline{UNKNOWN}.
\item RB-6: \uline{UNKNOWN}.
\item RB-7: \uline{UNKNOWN}.
\end{itemize}
\end{itemize}
\subsubsection{Low Voltage Interface}
\label{sec:org7913edb}
The readout board will connect to the low voltage supply to receive \textasciitilde{}8V power.
\begin{itemize}
\item The part number for the LV connector is \uline{UNKNOWN}.
\item The pinout for the LV connector is \uline{UNKNOWN}.
\item The placement for the LV connector is \uline{UNKNOWN}.
\end{itemize}
\subsubsection{Programming Interface}
\label{sec:orgbd5869c}
\begin{itemize}
\item The readout board will provide a programming interface to allow fusing/configuration of the lpGBT through I2C.
\item The part number for the programming connector is \uline{UNKNOWN}.
\item The pinout for the programming connector is \uline{UNKNOWN}.
\item The placement for the programming connector is \uline{UNKNOWN}.
\end{itemize}
\subsection{Signal Connectivity}
\label{sec:org3a33f1a}
\subsubsection{I2C}
\label{sec:org1ec0f1a}
\begin{itemize}
\item The GBT-SCA will provide one I2C connection for each module.
\item All ETROCs in a module will share an I2C master.
\item The readout board will provide strong I2C pull-ups.
\begin{itemize}
\item It is assumed that the modules will not, and have only weak pull-ups.
\end{itemize}
\end{itemize}
\subsubsection{IO}
\label{sec:orgceae83f}
\begin{itemize}
\item A GBT-SCA provides 32 tri-stateable 1.5V GPIO
\item An LPGBT provides 16 tri-stateable 1.2V GPIO
\item These IO will be allocated as:
\begin{center}
\begin{tabular}{llrrr}
 &  & RB-3 & RB-6 & RB-7\\
\hline
GBT-SCA RESETB &  & 1 & 1 & 1\\
VTRX LD\_RESETB &  & 1 & 1 & 1\\
VTRX LD\_DIS &  & 1 & 1 & 1\\
Module Reset &  & 3 & 6 & 7\\
PB PGOOD &  & \uline{UNKNOWN} & \uline{UNKNOWN} & \uline{UNKNOWN}\\
PB EN &  & \uline{UNKNOWN} & \uline{UNKNOWN} & \uline{UNKNOWN}\\
\end{tabular}
\end{center}
\end{itemize}

\subsubsection{Uplinks}
\label{sec:org56e9832}
\begin{itemize}
\item From every module, the readout board will receive:
\begin{itemize}
\item RB-3: \uline{UNKNOWN} uplinks
\item RB-6: \uline{UNKNOWN} uplinks
\item RB-7: \uline{UNKNOWN} uplinks
\end{itemize}
\item These uplinks will operate at up to:
\begin{itemize}
\item RB-3: \uline{UNKNOWN} Mbps
\item RB-6: \uline{UNKNOWN} Mbps
\item RB-7: \uline{UNKNOWN} Mbps
\end{itemize}
\item These uplinks will \textbf{not} be phase length matched.
\end{itemize}
\subsubsection{Downlinks}
\label{sec:org4ecae42}
\begin{itemize}
\item The readout board will deliver to each module two downlinks that provide a ``fast command'' interface to the ETROC.
\item These fast-command downlinks will run at \uline{UNKNOWN} Mbps.
\item The fast command downlinks will be multi-dropped on the module, with each downlink serving 2 ETROCs.
\begin{itemize}
\item 6 downlinks for the RB-3, 12 downlinks for the RB-6, and 14 downlinks for the RB-7.
\end{itemize}
\item These downlinks will \textbf{not} be phase length matched \emph{between} modules, but \textbf{will} be phase length matched \emph{within} modules and to their respective clocks.
\end{itemize}
\subsubsection{Clocking}
\label{sec:org77f30c7}
\begin{itemize}
\item The readout board will be responsible for delivering a 40 MHz point-to-point clock to each of the ETROCs it connects to.
\begin{itemize}
\item 12 clocks for the RB-3, 24 clocks for the RB-6, and 28 clocks for the RB-7.
\end{itemize}
\item These clocks will \textbf{not} be phase length matched \emph{between} modules, but \textbf{will} be phase length matched \emph{within} modules.
\item The clock will be distributed only from the \textbf{master} lpGBT; the slave lpGBT clock outputs will not be used due to radiation intolerance.
\end{itemize}
\subsubsection{VTRX}
\label{sec:org1a567ec}
\subsection{Monitoring}
\label{sec:org0073a70}

A GBT-SCA ASIC provides 31 analog inputs with 12-bit resolution, and 4 analog outputs with 8-bit resolution, with a range of 0 to 1V.

An lpGBT provides 8 analog inputs with 10-bit resolution and 1 analog output with 12-bit resolution.

The readout board will monitor the following analog channels:

\subsubsection{Voltage Monitoring}
\label{sec:org62a8231}
\begin{center}
\begin{tabular}{llllllrrr}
 & Type & Monitored By & Divider & LSB & Range & Qty. RB-3 & Qty. RB-6 & Qty. RB-7\\
\hline
Sensor BV & Voltage & GBT-SCA & 82/100082 & 2.980e-01 V & 1220.3 V & \uline{UNKNOWN} & \uline{UNKNOWN} & \uline{UNKNOWN}\\
ETROC +1.2V & Voltage & GBT-SCA & 2.0/4.2 & 5.128e-04 V & 2.1 V & \uline{UNKNOWN} & \uline{UNKNOWN} & \uline{UNKNOWN}\\
PB +1.2V & Voltage & GBT-SCA & 2.0/4.2 & 5.128e-04 V & 2.1 V & \uline{UNKNOWN} & \uline{UNKNOWN} & \uline{UNKNOWN}\\
RB LV & Voltage & GBT-SCA & 1/11 & 2.686e-03 V & 11.0 V & 1 & 1 & 1\\
VTRX +2.5V RX & Voltage & lpGBT & 1.5/4.5 & 2.933e-03 V & 3.0 V & 1 & 1 & 1\\
VTRX +2.5V TX & Voltage & lpGBT & 1.5/4.5 & 2.933e-03 V & 3.0 V & 1 & 1 & 1\\
GBTX +1.5VD & Voltage & lpGBT & 2.0/3.5 & 1.711e-03 V & 1.8 V & 1 & 1 & 1\\
GBTX +1.5VA & Voltage & lpGBT & 2.0/3.5 & 1.711e-03 V & 1.8 V & 1 & 1 & 1\\
VTRX Temp & Temperature & lpGBT &  &  &  & 1 & 1 & 1\\
RB Temp & Temperature & GBT-SCA &  &  &  & 1 & 1 & 1\\
PB Temp & Temperature & \uline{UNKNOWN} &  &  &  & \uline{UNKNOWN} & \uline{UNKNOWN} & \uline{UNKNOWN}\\
Module Temp & Temperature & \uline{UNKNOWN} &  &  &  & \uline{UNKNOWN} & \uline{UNKNOWN} & \uline{UNKNOWN}\\
VTRX RSSI & Photocurrent & lpGBT &  &  &  & 1 & 1 & 1\\
\end{tabular}
\end{center}

\begin{itemize}
\item All voltage dividers will be formed of 0.5\% tolerance resistors.
\item All voltage dividers will be decoupled by 0.1 uF capacitors.
\item The ETROC voltage monitors assume that a 1.2k resistor is in series with the \texttt{1V2\_MON} signal from the module.
\end{itemize}

\subsection{Low Voltage Distribution}
\label{sec:org08aba4a}
\begin{itemize}
\item The readout board will provide four 47 uF capacitors connected to each 1.2V ETROC supply.
\begin{itemize}
\item There will be no additional filtering.
\end{itemize}
\item Analog and digital power for the ETROC will not be distinguished.
\item The low voltage will be ganged such that \uline{UNKNOWN} ETROCs share a common power supply.
\end{itemize}
\subsection{Bias Voltage Distribution}
\label{sec:orga9a4306}
\begin{itemize}
\item Bias voltage will be a maximum of \uline{UNKNOWN} volts.
\item The bias voltage granularity will be:
\begin{itemize}
\item \uline{UNKNOWN} channels for an RB-3
\item \uline{UNKNOWN} channels for an RB-6
\item \uline{UNKNOWN} channels for an RB-7
\end{itemize}
\item The readout board will provide a filter for each bias voltage channel consisting of a 200 ohm resistor and 1500 pF capacitor, which will be rated for at least 1000V.
\end{itemize}
\subsection{Mechanics}
\label{sec:org1bfe7bc}
\subsubsection{Outer Dimensions}
\label{sec:org7e690de}
\begin{itemize}
\item The outer dimension of the readout board will follow an \uline{UNKNOWN} shape
\end{itemize}
\subsubsection{Screw Holes \& Sizes}
\label{sec:org3739daf}
\begin{itemize}
\item The readout board will have \uline{UNKNOWN} mounting holes of size \uline{UNKNOWN} in the following locations:
\begin{enumerate}
\item \uline{UNKNOWN}
\end{enumerate}
\end{itemize}
\subsubsection{Thickness}
\label{sec:orgfe1b549}
\begin{itemize}
\item The readout board will be 1.0mm thick with a manufacturing specification of \textpm{} 10\%.
\end{itemize}
\subsection{Component List}
\label{sec:org7651943}
\end{document}
