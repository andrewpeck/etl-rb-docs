% Created 2022-02-21 Mon 09:55
% Intended LaTeX compiler: pdflatex
\documentclass[11pt]{article}
\usepackage[utf8]{inputenc}
\usepackage[T1]{fontenc}
\usepackage{fixltx2e}
\usepackage{fullpage}
\usepackage{graphicx}
\usepackage{longtable}
\usepackage{float}
\usepackage{wrapfig}
\usepackage{rotating}
\usepackage[normalem]{ulem}
\usepackage{amsmath}
\usepackage{textcomp}
\usepackage{marvosym}
\usepackage{wasysym}
\usepackage{amssymb}
\usepackage{hyperref}
%\usepackage{mathpazo}
\renewcommand{\familydefault}{\sfdefault}
\usepackage{color}
\usepackage{enumerate}
\definecolor{bg}{rgb}{0.95,0.95,0.95}
\tolerance=1000

\linespread{1.1}
\hypersetup{pdfborder=0 0 0}
\author{Andrew Peck, Daniel Spitzbart}
\date{\today}
\title{Module PCB Specifications}
\hypersetup{
 pdfauthor={Andrew Peck, Daniel Spitzbart},
 pdftitle={Module PCB Specifications},
 pdfkeywords={},
 pdfsubject={},
 pdfcreator={Emacs 28.0.91 (Org mode 9.5.2)}, 
 pdflang={English}}
\begin{document}

\maketitle
\section{Module Specifications}
\label{sec:org608ffa0}

\begin{itemize}
\item Authors: Andrew Peck, Daniel Spitzbart

\item Modification Date: 2022-02-21 09:55

\item Status: This document is missing 42 pieces of information concerning  37 specifications
\begin{itemize}
\item Specifications are \textbf{-13.5\% complete}.
\end{itemize}

\item A pdf version of this document can be found \href{./module-specs.pdf}{here}. Please check the timestamp to ensure it is up to date. The master copy of this document is an emacs org mode file found \href{https://gitlab.cern.ch/cms-etl-electronics/readout-board-docs/-/blob/master/docs/Specifications/module-specs.org}{here}.
\end{itemize}

\setcounter{tocdepth}{3}
\tableofcontents

\section{Specifications}
\label{sec:org6bcb938}

\subsection{Description}
\label{sec:org631e6b5}

The module PCB is a simple printed circuit board which will host two LGAD sensors, and two LGAD modules. To support these, it will have wire bond pads, a variety of passive components, and a connector interface to attach to a Readout Board (RB).

\begin{itemize}
\item Documentation for the ETROC does not exist.
\end{itemize}

\subsection{Layout}
\label{sec:org73d1153}
We denote the "top" side of the PCB as that containing the module and BV connectors.

We denote the "bottom" side of the PCB as that containing the sensor.

\subsubsection{Sensor Placement}
\label{sec:org288d0c8}
\begin{itemize}
\item The dimensions of a Sensor+ETROC assembly is \uline{UNKNOWN} \texttimes{} \uline{UNKNOWN} mm.
\item The position of the two Sensor+ETROC assemblies are (positions relative to \uline{UNKNOWN}):
\begin{itemize}
\item x= \uline{UNKNOWN} mm;  y= \uline{UNKNOWN} mm.
\item x= \uline{UNKNOWN} mm;  y= \uline{UNKNOWN} mm.
\end{itemize}
\end{itemize}
\subsubsection{Wire bonding}
\label{sec:orgd302e17}
\begin{itemize}
\item A wire bonding diagram of an ETROC is shown in \uline{UNKNOWN}
\item The pad pitch will be \uline{UNKNOWN} mm.
\item The pad aperture will be \uline{UNKNOWN} \texttimes{} \uline{UNKNOWN} mm.
\item The pad aperture will be NSMD (non-solder-mask-defined) with the mask oversized by \uline{UNKNOWN} mm.
\item The wire bond pads will be located \uline{UNKNOWN} mm from the edge of the ETROC (measured edge-to-edge).
\end{itemize}
\subsubsection{Grounding}
\label{sec:org71da851}
\begin{itemize}
\item The RB will geometrically isolate analog and digital ground, with specified areas of the PCB filled by a digital ground pour, and others filled by an analog ground pour. The digital and analog grounds will be connected together at a single point.
\begin{itemize}
\item A drawing of the grounding scheme is shown in \uline{UNKNOWN}.
\end{itemize}
\end{itemize}
\subsubsection{Fiducial Markings}
\label{sec:orga2a40f7}
\begin{itemize}
\item The module PCB will have on its bottom side 4 fiducial markers, composed of circles with crosses through them. These markers shall be placed at:
\begin{enumerate}
\item \uline{UNKNOWN}
\item \uline{UNKNOWN}
\item \uline{UNKNOWN}
\item \uline{UNKNOWN}
\end{enumerate}
\end{itemize}
\subsection{Connectivity}
\label{sec:org85f2608}
\subsubsection{Readout Board Interface}
\label{sec:org15b99a9}
\begin{enumerate}
\item Signal Interface
\label{sec:orgd3a35d3}
The signal interface to the readout board will consist of:
\begin{itemize}
\item The readout board will use connector part number \uline{UNKNOWN}.
\item The pinout of the readout board connectors is \uline{UNKNOWN}.
\item The placement of these connectors is \uline{UNKNOWN}.
\end{itemize}
\item BV Interface
\label{sec:org9685554}
The BV interface to the readout board will consist of:
\begin{itemize}
\item The BV to readout board interface will use connector part number \uline{UNKNOWN}.
\item The pinout of these connectors is \uline{UNKNOWN}.
\item The placement of these connectors is \uline{UNKNOWN}.
\end{itemize}
\end{enumerate}
\subsubsection{I2C}
\label{sec:org7ab7be2}

\begin{itemize}
\item The module carries I2C signals (SCL, SCK) from the readout board and distributes it to the 4 ETROCs in a star topology.
\item The module PCB will provide independent I2C addresses for each ETROC on a module. Addresses will be 0/1/2/3 corresponding to the slot, and are set directly by wire bonds.
\begin{itemize}
\item Addresses \textbf{will not} be set by resistors, and can not be modified.
\end{itemize}
\item The module PCB will \textbf{not} provide pull-up resistors on I2C lines. These will be provided by the host-system.
\end{itemize}

\subsubsection{Low Voltage}
\label{sec:org12cc19b}

The module must receive +1.2V from the readout board, and distribute it to the ETROCs in a low inductance, low resistance path.

\begin{itemize}
\item Each module will receive two \emph{possibly} independent +1.2V supplies.
\begin{itemize}
\item They will not be connected together in any way on the module, but \emph{may} be ganged together on the RB.
\end{itemize}
\end{itemize}

\begin{enumerate}
\item Decoupling
\label{sec:org3c59432}
\begin{itemize}
\item The module will provide decoupling capacitors on the +1.2V supplies. The power filtering network will be composed of:
\begin{enumerate}
\item \uline{UNKNOWN} resistors of \uline{UNKNOWN} value
\item \uline{UNKNOWN} resistors of \uline{UNKNOWN} value
\end{enumerate}
\item Decoupling capacitors will be placed as close as possible to the ETROC, and follow standard practices to maintain low inductance connections.
\item Decoupling capacitors will be suitably rated to minimize DC bias effects.
\item To reduce temperature dependence, ceramics will be chosen where possible with minimal temperature dependence (e.g. X7R).
\end{itemize}
\item Monitoring
\label{sec:org4c7fdc1}
\begin{itemize}
\item The module will provide \textbf{two} feedback voltages for point-of-load monitoring. It will be delivered back to the RB through a 1.2k 0.5\% resistor. These point-of-load monitoring resistors will be placed close to each pair of ETROCs at their respective ends of the module.
\end{itemize}
\end{enumerate}
\subsubsection{Bias Voltage}
\label{sec:org2a58f71}

The module will receive bias voltage from the readout board and distribute it to the modules.

\begin{itemize}
\item BV will be a maximum of \uline{UNKNOWN} volts.
\item There will be \uline{UNKNOWN} bias voltage supplies for each module.
\end{itemize}
\begin{enumerate}
\item Decoupling
\label{sec:org595b90e}
\begin{itemize}
\item The BV may or may not be decoupled/filtered on the module PCB \uline{UNKNOWN}
\end{itemize}
\end{enumerate}
\subsubsection{Signal Connectivity}
\label{sec:orge7a7477}
\begin{itemize}
\item Each module will receive two \uline{UNKNOWN} MHz downlinks from the RB
\item Each module will receive four 40 MHz clocks from the RB
\begin{itemize}
\item The clocks shall be length matched and skewed such that for a multi-drop pair of lpGBTs, the clock and data are synchronized at each ETROC's input pads.
\end{itemize}
\item Each module will have \uline{UNKNOWN} uplinks operating at up to \uline{UNKNOWN} Mbps.
\item The module will host \uline{UNKNOWN} temperature sensors, which will be monitored in the RB.
\end{itemize}
\subsection{Mechanics}
\label{sec:org81f4448}
\subsubsection{Outer Dimensions}
\label{sec:org43a31b1}
\begin{itemize}
\item The outer dimension of the Module PCB will follow a rectangular shape, with dimensions of \uline{UNKNOWN} \texttimes{} \uline{UNKNOWN}.
\end{itemize}
\subsubsection{Screw Holes \& Sizes}
\label{sec:org5e11710}
\begin{itemize}
\item The Module PCB will have \uline{UNKNOWN} mounting holes of size \uline{UNKNOWN} in the following locations:
\begin{enumerate}
\item \uline{UNKNOWN}
\end{enumerate}
\end{itemize}
\subsubsection{Thickness}
\label{sec:org4608cd5}
\begin{itemize}
\item The Module PCB will be 0.5mm thick with a manufacturing specification of \textpm{} 10\%.
\end{itemize}
\subsubsection{Drawings}
\label{sec:orgfcad6f2}
A drawing of the Module PCB is available at \uline{UNKNOWN}.
\subsubsection{Mechanical Interface}
\label{sec:orgbec9825}
\begin{itemize}
\item the module shall be aligned to the Readout Board using an \uline{UNKNOWN} keying mechanism
\end{itemize}
\end{document}